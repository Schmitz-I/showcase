% Options for packages loaded elsewhere
\PassOptionsToPackage{unicode}{hyperref}
\PassOptionsToPackage{hyphens}{url}
%
\documentclass[
]{article}
\usepackage{amsmath,amssymb}
\usepackage{iftex}
\ifPDFTeX
  \usepackage[T1]{fontenc}
  \usepackage[utf8]{inputenc}
  \usepackage{textcomp} % provide euro and other symbols
\else % if luatex or xetex
  \usepackage{unicode-math} % this also loads fontspec
  \defaultfontfeatures{Scale=MatchLowercase}
  \defaultfontfeatures[\rmfamily]{Ligatures=TeX,Scale=1}
\fi
\usepackage{lmodern}
\ifPDFTeX\else
  % xetex/luatex font selection
\fi
% Use upquote if available, for straight quotes in verbatim environments
\IfFileExists{upquote.sty}{\usepackage{upquote}}{}
\IfFileExists{microtype.sty}{% use microtype if available
  \usepackage[]{microtype}
  \UseMicrotypeSet[protrusion]{basicmath} % disable protrusion for tt fonts
}{}
\makeatletter
\@ifundefined{KOMAClassName}{% if non-KOMA class
  \IfFileExists{parskip.sty}{%
    \usepackage{parskip}
  }{% else
    \setlength{\parindent}{0pt}
    \setlength{\parskip}{6pt plus 2pt minus 1pt}}
}{% if KOMA class
  \KOMAoptions{parskip=half}}
\makeatother
\usepackage{xcolor}
\usepackage[margin=1in]{geometry}
\usepackage{graphicx}
\makeatletter
\def\maxwidth{\ifdim\Gin@nat@width>\linewidth\linewidth\else\Gin@nat@width\fi}
\def\maxheight{\ifdim\Gin@nat@height>\textheight\textheight\else\Gin@nat@height\fi}
\makeatother
% Scale images if necessary, so that they will not overflow the page
% margins by default, and it is still possible to overwrite the defaults
% using explicit options in \includegraphics[width, height, ...]{}
\setkeys{Gin}{width=\maxwidth,height=\maxheight,keepaspectratio}
% Set default figure placement to htbp
\makeatletter
\def\fps@figure{htbp}
\makeatother
\setlength{\emergencystretch}{3em} % prevent overfull lines
\providecommand{\tightlist}{%
  \setlength{\itemsep}{0pt}\setlength{\parskip}{0pt}}
\setcounter{secnumdepth}{-\maxdimen} % remove section numbering
\ifLuaTeX
  \usepackage{selnolig}  % disable illegal ligatures
\fi
\usepackage{bookmark}
\IfFileExists{xurl.sty}{\usepackage{xurl}}{} % add URL line breaks if available
\urlstyle{same}
\hypersetup{
  pdftitle={Final Project, STAT/Q Sci 403},
  pdfauthor={Ilse Schmitz},
  hidelinks,
  pdfcreator={LaTeX via pandoc}}

\title{Final Project, STAT/Q Sci 403}
\author{Ilse Schmitz}
\date{June 04 2025}

\begin{document}
\maketitle

\subsection{Introduction}\label{introduction}

\section{Model Selection}\label{model-selection}

In this case, I decided to use AIC to help select which parameters to
use, however this is the wrong way to select features. I chose AIC since
the added flexibility is preferable to a smaller dataset. Whether the
training data set is small may be up to interpretation, but given it's
context in real estate, I've deemed the dataset small given the vast
amount of real estate data that exists in the world.

\begin{verbatim}
##   (Intercept)         price      bedrooms     bathrooms   sqft_living 
## -1.012988e+01  3.254803e-07  1.462580e-02  3.442956e-03 -1.786466e-03 
##      sqft_lot        floors    waterfront          view     condition 
##  2.082069e-06  5.017800e-02  1.511741e-01 -1.160520e-02  2.692756e-02 
##         grade    sqft_above sqft_basement      yr_built  yr_renovated 
##  6.211073e-02  1.730939e-03  1.797815e-03 -1.014626e-03  3.562950e-05 
##       zipcode sqft_living15    sqft_lot15 
##  1.720314e-04  7.498896e-05 -2.298239e-06
\end{verbatim}

Based on the information here, and using AIC as our criteria, from
forward selection we end up with 11 predictors,~ (grade, yr\_built,
sqft\_living15, sqft\_lot15, floors, bedrooms, sqft\_lot, condition,
yr\_renovated, zipcode, sqft\_above)

and with backward selection we end up with 13 predictors,~ (grade,
yr\_built, sqft\_living15, sqft\_lot15, floors, bedrooms, sqft\_lot,
condition, yr\_renovated, zipcode, sqft\_living, view, sqft\_basement)

However, as stated previously, this is the wrong way to select the
variables in this case. As such, we use prior knowledge to estimate
which features to use in our regression and prediction tasks.

By constructing a series of graphs comparing the log10price and price to
all other features, I selected features to utilize based on appearence
and whether there appeared to be a relationship between the two. the
features I chose to keep were:

sqft\_living15

sqft\_above

grade

floors

sqft\_living

bathrooms

bedrooms

The features I chose to discard were:

sqft\_lot15

zipcode

yr\_renovated

yr\_built

condition

waterfront

sqft\_lot

sqft\_basement

view

These graphs can be found at the bottom of this report, in order to not
impede reading, as there are 32 in total.

\pagebreak

\section{Regression Task}\label{regression-task}

\begin{verbatim}
##       5%      95% 
## 4.704351 4.906639 
##            5%           95% 
## -1.604774e-06  6.162290e-05 
##            5%           95% 
## -7.730599e-05  5.372580e-07 
##        5%       95% 
## 0.0761488 0.1106130 
##           5%          95% 
## -0.063306602  0.002099521 
##           5%          95% 
## 5.545086e-05 1.382196e-04 
##          5%         95% 
## -0.02220956  0.03657248 
##          5%         95% 
## -0.02007013  0.01630754
\end{verbatim}

\includegraphics{Project_template_files/figure-latex/unnamed-chunk-3-1.pdf}
\includegraphics{Project_template_files/figure-latex/unnamed-chunk-3-2.pdf}

For computing the Gaussian interval, two plots have been provided, one
including the intercept, and one without, in order to better interpret
the differences between the confidence intervals and the regression
coefficients. Blue is the Gaussian confidence intervals, and red is the
bootstrap. Based on these graphs, we can see the bedrooms and floors
features seem to have the most variation, with grade and floors second.
All three features dealing with square footage have shockingly small
confidence intervals, implying very high correlation between price and
square footage of at least the three different types included within
this model.

For the Bootstrap model, our confidence intervals follow very similar
patterns, with the only notable exception being the interval of the
bedroom and bathroom coefficients being distinctly larger. Additionally,
much like the Gaussian CI's, there is a strong confidence in the square
footage features and their ability to predict price, which lends
credence to those variables being the best indicators.

\pagebreak

\section{Conformal Prediction (Jackknife+)
Task}\label{conformal-prediction-jackknife-task}

\includegraphics{Project_template_files/figure-latex/unnamed-chunk-4-1.pdf}

For the Guess.dat, the MSE for test\_1 was based on the MSE of the
training data when sent through the regression model, using both the
price and the log10price. Based on prior experience, if the model has
been correctly set up test sets usually score a higher MSE than ther
training counterparts, but it does tend to be close. As such, my guess
for the MSE of the log10price is 0.9, rounding up the existing MSE from
the training data.

\begin{verbatim}
## [1] 57726344874
\end{verbatim}

\begin{verbatim}
## [1] 0.08743939
\end{verbatim}

We know the MSE of a test set is very likely to be larger than the MSE
of the training set, and based on this, I would assume the MSE is around
60,000,000,000, close to the MSE of the training set, but slightly
larger.~ For the confidence interval, We can use the coverage of the
test\_2 set as a basis for the true coverage. I would suspect there will
be some variability, but for the sake of a clean guess, I would be the
true coverage would be around 0.935, to match the existing prediction
data.

\begin{verbatim}
## [1] 0.935
\end{verbatim}

\subsubsection{Additional graphs}\label{additional-graphs}

\includegraphics{Project_template_files/figure-latex/unnamed-chunk-7-1.pdf}
\includegraphics{Project_template_files/figure-latex/unnamed-chunk-7-2.pdf}
\includegraphics{Project_template_files/figure-latex/unnamed-chunk-7-3.pdf}
\includegraphics{Project_template_files/figure-latex/unnamed-chunk-7-4.pdf}
\includegraphics{Project_template_files/figure-latex/unnamed-chunk-7-5.pdf}
\includegraphics{Project_template_files/figure-latex/unnamed-chunk-7-6.pdf}
\includegraphics{Project_template_files/figure-latex/unnamed-chunk-7-7.pdf}
\includegraphics{Project_template_files/figure-latex/unnamed-chunk-7-8.pdf}

\end{document}
